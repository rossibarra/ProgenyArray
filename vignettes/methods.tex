\documentclass[11pt]{article}
\usepackage{fullpage}
\usepackage{amsmath,amssymb,amsthm}
\usepackage{graphicx}
\usepackage[authoryear,round,longnamesfirst]{natbib}
\usepackage[x11names, rgb]{xcolor}
\usepackage[utf8]{inputenc}
\usepackage{tikz}
\usetikzlibrary{snakes,arrows,shapes}
\usepackage{hyperref}
\hypersetup{
colorlinks=true
}

\renewcommand{\P}{\mathbb{P}}
\newcommand{\E}{\mathbb{E}}
\DeclareMathOperator{\var}{var}
\DeclareMathOperator{\cov}{cov}

\title{
Progeny Array Statistical Genetics Methods
}

\author{
Vince Buffalo
}

\begin{document}
\maketitle

\section{Parentage Inference Methods}

Parentage inference methods are inspired by those from \citet{meagher1986},
which are reviewed in \citet{marshall1998}. The major difference in our model
is the incorporation of an error model that allows for different error rates
for heterozygous and homozygous genotypes. We'll step through Meagher and
Thompson's original model before deriving the model with error.

\subsection{Original Model (without Genotyping Error)}

\citeauthor{meagher1986}'s basic model is to evaluate the relatedness of three
individuals, $M$, $F$, and $O$, using a likelihood calculated from the joint
probability of these individuals' genotypes. Here, $M$, $F$, and $O$ represent
alleged mother, alleged father, and offspring, respectively. For each loci, we
calculate the probability of these individuals' genotypes $g_m$, $g_f$, and
$g_o$ (respectively) under two pedigree models: that $M$ and $F$ are the
parents of $O$, and that $M$, $F$, and $O$ are all unrelated individuals.
Following Meagher and Thompson's notation, these two relationships are denoted
$QQ$ and $UU$. Another possible relationship between these individuals is that
neither $M$ or $F$ is a parent of $O$; we denote this as $QU$ (we'll skip
including this possibility for now). Thus, for a single locus the joint
probability of genotypes $g_m$, $g_m$, and $g_o$ given that individuals $M$ and
$F$ are $O$'s parents is:

$$ P(g_m, g_f, g_o | QQ) = T(g_o | g_m, g_f) P(g_m) P(g_f) $$

(from \citealt{meagher1986}). Here, $T(g_o | g_m, g_f)$ is the
Mendelian transmission matrix --- essentially a conditional probability matrix.
$P(g_m)$ and $P(g_f)$ are the probability of the observed genotypes for mother
and father. Assuming random mating, these are just the Hardy-Weinberg genotype
probabilities. For any genotype with alternate allele frequency $p$, the vector
$\mathbf{h}$ encodes the genotype probabilities under Hardy-Weinberg:

$$
\mathbf{h} = \left[ \begin{array}{c}
(1 - p)^2 \\
2p(1-p) \\
p^2 \\
\end{array} \right]
$$

\subsection{Genotype Likelihoods with Error}

At this point, this model assumes $g_m$, $g_f$, and $g_o$ are known with
certainty. In reality, the true genotypes $g_m$, $g_f$, and $g_o$ are hidden
and only the genotypes with error $g_m'$, $g_f'$, and $g_o'$ are observed.
Graphically, this is depicted in \autoref{fig:dag}.

\begin{figure}
\centering
\begingroup
\tikzset{every picture/.style={scale=0.8}}%
% dot -Txdot graphmodel.dot | dot2tex -tmath --figpreamble="\Large" --figonly > graphmodel.tex
\input{graphmodel.tex}
\endgroup

\caption{A directed graph model with hidden genotype variables $g_m$, $g_f$,
and $g_o$, and observed genotypes $g_m'$, $g_f'$, and $g_o'$.}

\label{fig:dag}

\end{figure}

To model genotyping errors, we derive an expression for $P(g_m', g_f', g_o')$
(given $QQ$) that marginalizes over the hidden genotype variables $g_m$, $g_f$,
and $g_o$. So,

$$
P(g_m', g_f', g_o') = \sum_{g_m,g_f,g_o \in \Omega} P(g_m', g_f', g_o' | g_m, g_f, g_o) P(g_m, g_f, g_o)
$$

The advantage of this approach is that we know through Mendelian transmission
and Hardy-Weinberg that the component $P(g_m, g_f, g_o) = T(g_o | g_m, g_f)
P(g_m) P(g_f)$, so,

$$
P(g_m', g_f', g_o') = \sum_{g_m,g_f,g_o \in \Omega} P(g_m', g_f', g_o' | g_m, g_f, g_o) T(g_o | g_m, g_f) P(g_m) P(g_f)
$$

Now, $P(g_m', g_f', g_o' | g_m, g_f, g_o)$ seems complex, but each of these
observed genotypes conditioned on the real genotype is independent of the
others. For example, $g_o' \perp\!\!\!\perp g_m'\; | \;g_m$, meaning that
knowing $g_m'$ doesn't provide any information about the observed offspring's
genotype $g_o'$ given that we know $g_m$. This allows us to say $P(g_m', g_f',
g_o' | g_m, g_f, g_o) = P(g_m' | g_m) P(g_f'|g_f) P(g_o'|g_o)$. So:

$$
P(g_m', g_f', g_o') = \sum_{g_m,g_f,g_o \in \Omega} P(g_m' | g_m) P(g_f'|g_f) P(g_o'|g_o) T(g_o | g_m, g_f) P(g_m) P(g_f)
$$

%Rearranging to make this equation's connection with the graph model clear:
%
%$$
%P(g_m', g_f', g_o') = \sum_{g_m,g_f,g_o \in \Omega} P(g_o'|g_o) T(g_o | g_m, g_f) P(g_m) P(g_m' | g_m) P(g_f) P(g_f'|g_f)
%$$
%
Every component of this model is now tractable. Our error model enters through
the stochastic transition matrix $\mathbf{E}$ for any true, latent genotype $g$
to an observed genotype $g'$:

$$
\mathbf{E} = P(g'|g) = \left[ \begin{array}{ccc}
1 - e & e/2 & e/2 \\
\epsilon/2 & 1 - \epsilon & \epsilon/2 \\
e/2 & e/2 & 1 - e \\
\end{array} \right]
$$

Where $e$ is the homozygous error rate and $\epsilon$ is the heterozygous error
rate. This error model is similar to models in which the probability of error
is distributed uniformly over the two erroneous genotypes
\citep*{sobel2002detection,lincoln1992systematic}.

Extending this to all loci is trivial; we assume independence across loci, so
for all loci we sum the log probability. Allowing locus $l$'s joint probability
to be written as $P(g_{m,l}', g_{f,l}', g_{o,l}')$, the probability of $M$ and
$F$ being $O$'s parents is:

$$
P(G_m', G_f', G_o' | QQ) = \sum_{l \in L} \log P(g_{m,l}', g_{f,l}', g_{o,l}')
$$

%
%The sum is taken over all possible combinations of $g_m$, $g_f$, $g_o$. Due to
%exclusion some of these have zero probability. Since we assume biallelic loci,
%there are three possible genotypes at each locus, so this is a sum of $3^3 =

%27$ terms. However, due to exclusion twelve of these transmission
%probabilities are zero (e.g. when mother is 0/0 and father is 1/1, progeny
%must be 0/1).
%
%
\subsection{Parental Genotype Imputation}

\begin{figure}
\centering
\begingroup
\tikzset{every picture/.style={scale=0.6}}%
% dot -Txdot parent-imput.dot | dot2tex -tmath --figpreamble="\Large" --figonly > parent-imput.tex
\input{parent-imput.tex}
\endgroup

\caption{A half-sib (with shared mother) with five offspring.}

\label{fig:dag}

\end{figure}

$$
P(g_m, g_{o,.}') = P(g_m) \prod_{i = 1}^n P(g_{o,i}' | g_m)
$$

$$
P(g_m, g_{o,.}') = P(g_m) \prod_{i = 1}^n \sum_{g_o \in \Omega} P(g_{o,i}', g_{o,i} | g_m)
$$

$$
P(g_m, g_{o,.}') = P(g_m) \prod_{i = 1}^n \sum_{g_o \in \Omega} P(g_{o,i}', g_{o,i} | g_m)
$$

%$$
%P(g_m, g_{o,.}') = P(g_m) P(g_{o,.}' | g_m)
%$$

\begin{equation}
  P(g_m, g_{o,.}') = P(g_m) \prod_{i = 1}^n \sum_{g_o \in \Omega} P(g_{o,i}' | g_{o,i}) P(g_{o,i} | g_{m})
\end{equation}

The form of this model is similar to a naive Bayesian model
(\citealt{koller2009}, p. 50), with an added layer of uncertainty from true
offspring genotypes to observed offspring genotypes.

%
%\begin{equation*}
%  P(g_m | g_o{o,.}') = \frac{P(g_m, g_{o,.}')}{P(g_{o,.}')}
%\end{equation*}
%

\subsection{Parental Haplotype Inference}



\bibliographystyle{plainnat}
\bibliography{methods}

\end{document}
